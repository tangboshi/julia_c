% !TeX document-id = {c8d06508-e420-4331-a32a-ee892ed71c0e}
% !TeX TXS-program:compile = txs:///pdflatex/[--shell-escape]

\documentclass{article}
\usepackage{minted}

\usepackage[utf8]{inputenc}
\usepackage[T1]{fontenc}
\usepackage{color}
\usepackage{soul}
\usepackage{amsmath}
\usepackage{amssymb}
\usepackage{listings}
\usepackage{minted}
\usepackage{hyperref}
\usepackage{graphicx}
\usepackage{calc}
\usepackage{enumitem}
\usepackage{standalone}

\begin{document}

\title{Yay, mein erstes C-Programm}
\author{Alexander Pastor}
\date{07.05.2017}
\maketitle
\tableofcontents
\newpage

\section{Ein paar Befehle mit der Konsole}

\begin{description}[leftmargin=!, labelwidth=1.7in]
	\item[ls] Zeigt den Verzeichnisinhalt an.
	\item[cd Ordner] Öffnet einen Ordner des Namens Ordner.
	\item[mkdir MeinOrdner] Erstellt ein Verzeichnis mit dem Namen MeinOrdner
	\item[rm Datei] Löscht eine Datei.
	\item[rm -r Ordner] Löscht einen Ordner (-r heißt rekursiv).
	\item[sudo Befehl] Vor einem Befehl vorangestellt, wird der Befehl mit Adminrechten durchgeführt.
	\item[man Befehl] Schlägt im Handbuch den Befehl nach.
	\item[touch Datei] "Berührt" eine Datei. Falls die Datei nicht existiert, wird sie erstellt.
	\item[mv Quelle Ziel]  Verschiebt eine Datei von Quelle nach Ziel wenn Pfad als Ziel angegeben ist, ansonsten wird die Datei in Ziel umbenannt.
	\item[. .. $\sim$ /] . steht für das aktuelle Verzeichnis, .. für das Vaterverzeichnis, ~ für das Homeverzeichnis des Benutzers und / für das Wurzelverzeichnis. Für den Administrator sind $\sim$ und / gleichbedeutend.
	\item[su] Sich als Administrator einloggen (nicht benutzen! - ein vorangestelltes sudo vor einzelnen Befehlen funktioniert ganz gut und ist kein permanentes Sicherheitsrisiko.)
	\item[Tab] Mit Tab kann ich die Befehle automatisch vervollständigen lassen. Einfach mal bei jeder Gelegenheit Tab drücken und gucken was passiert ;-)
\end{description}

\section{Mein erstes C-Programm}
\subsection{Vorbereitungen}

Wir öffnen das Terminal und dann tippen wir:

\begin{minted}[tabsize=4]{bash}
# Erst in das Verzeichnis deiner Wahl begeben
# zB ins Homeverzeichnis
cd ~
# Dann halten wir ein wenig Ordnung und erstellen ein Verzeichnis.
mkdir 2017_Arduino
cd 2017_Arduino
mkdir Hello_World
cd Hello_World
# Man beachte die Veränderung zwischen dem ersten und zweiten ls
ls
touch hello.c
ls
nano hello.c
\end{minted}

Yay, wir haben unsere erste eigene Datei erstellt. Mit der Konsole! Und dann bearbeiten wir sie auch noch mit der Konsole... \#HackerLifestyle

\bigskip

Jetzt tippen wir einfach folgendes ein (jap GENAU so):

\begin{minted}[tabsize=4]{cpp}
#include <stdio.h>

int main()
{
	printf("Hello World!\n\n");
	return 0;
}
\end{minted}

Wir speichern das ganze mit einem schnellen Cmd+O ab und verlassen das Programm mit Cmd+X.

\bigskip

Jetzt noch schnell dem Compiler Bescheid geben und voilà, c'est ça:

\begin{minted}{bash}
# -o steht für output und dahinter schreiben wir den Namen unseres tollen Programms
# Keine Sorge, das muss man nicht immer so machen, aber fürs erste ist es ganz hilfreich,
# wenn du den ganz ganz ganz oldschoolen Weg kennenlernst ;) 
# Der funktioniert immer, wahrscheinlich sogar auf einem grafikfähigen Taschenrechner ;-)
gcc hello.c -o hello

# Aus Sicherheitsgründen sind Programme nicht sofort ausführbar.
# Ändern wir das mal für unseres.
chmod +x ./hello

#Jetzt führen wir das Programm aus!!
./hello
\end{minted}

Der Rest von meiner C-Einführung sollte ganz analog funktionieren und ist "Hausaufgabe". Du kriegst das sicher hin. Bei Fragen einfach kurz schreiben.

\end{document}